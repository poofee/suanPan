\documentclass[5p]{elsarticle}
\usepackage{mathpazo,amsmath,amsfonts,amssymb,siunitx}
\bibliographystyle{elsarticle-num-names}
\newcommand*{\md}[1]{\mathrm{d}#1}
\newcommand*{\mT}{\mathrm{T}}
\newcommand*{\eqsref}[1]{Eq.~(\ref{#1})}
\newcommand*{\norm}[1]{\left\lVert#1\right\rVert}
\newcommand*{\pdfrac}[2]{\dfrac{\partial#1}{\partial#2}}
\newcommand*{\ddfrac}[2]{\dfrac{\md#1}{\md#2}}
\newdefinition{rmk}{Remark}
\ead{lei.zhang@pg.canterbury.ac.nz}
\begin{document}
\section{Basis}
The stress $\sigma$ can be decomposed into two part,
\begin{gather*}
\sigma=\left(1-d\right)\bar{\sigma}=\left(1-d\right)E\left(\varepsilon-\varepsilon^p\right).
\end{gather*}
The damage factor $d$ controls the degradation part and the effective stress $\bar{\sigma}$ represents the plasticity part thus can be expressed as $E\left(\varepsilon-\varepsilon^p\right)$.

For a typical iteration, the current strain $\varepsilon_n$ and current plastic strain $\varepsilon^p_n$ are known, it is asked to solve for the new stress estimation for a given strain increment $\Delta\varepsilon$. So it is legal to write
\begin{equation}
\begin{split}
\bar{\sigma}_{n+1}&=E\left(\varepsilon_{n+1}-\varepsilon_{n+1}^p\right)\\&=E\left(\varepsilon_{n}+\Delta\varepsilon-\varepsilon_n^p-\Delta\varepsilon^p\right)\\&=\sigma^{tr}-\Delta\sigma,
\end{split}
\end{equation}
with $\sigma^{tr}=E\left(\varepsilon_{n+1}-\varepsilon_n^p\right)$ is a fixed term for a given iteration and is called the elastic predictor. Furthermore, by decomposing the increment of plastic strain $\Delta\varepsilon^p$ into spherical and deviatoric part, the stress corrector $\Delta\sigma$ can be expressed as
\begin{gather*}
\Delta\sigma=E\Delta\varepsilon^p=2G\Delta{}e^p+K\Delta\theta^pI,
\end{gather*}
in which $G$ is the shear modulus, $K$ is the bulk modulus, $I$ is the second order unit tensor, $e^p$ is the deviatoric plastic strain and $\theta^p$ is the volumetric plastic strain.

Additionally, with the assist of the flow rule used,
\begin{gather*}
\Delta{}e^p=\dfrac{\bar{s}}{\norm{\bar{s}}}\Delta\lambda,\qquad
\dfrac{\Delta\theta^p}{3}=\alpha_p\Delta\lambda,
\end{gather*}
and $\Delta\lambda$ is the increment of the consistency parameter.

Similarly, the effective stress can also be decomposed into spherical and deviatoric part, that is
\begin{gather*}
\bar{s}+\dfrac{\bar{I}_1}{3}I=s^{tr}+\dfrac{I_1^{tr}}{3}I-\Delta\lambda\left(2G\dfrac{\bar{s}}{\norm{\bar{s}}}+3K\alpha_pI\right),
\end{gather*}
correspondingly,
\begin{gather*}
\bar{s}=s^{tr}-2G\dfrac{\bar{s}}{\norm{\bar{s}}}\Delta\lambda,\qquad
\bar{I}_1=I_1^{tr}-9K\alpha_p\Delta\lambda,
\end{gather*}
where $I_1$ is the first invariant of the stress tensor.

Computing the norm of the deviatoric part gives
\begin{gather*}
\norm{\bar{s}}=\norm{s^{tr}}-2G\Delta\lambda.
\end{gather*}
The above equations holds because that $\bar{s}$ and $\dfrac{\bar{s}}{\norm{\bar{s}}}$ are of the same direction, hence the norm is simply the summation of two components. Substituting it back gives
\begin{gather*}
\dfrac{\bar{s}}{\norm{\bar{s}}}=\dfrac{s^{tr}}{\norm{s^{tr}}}.
\end{gather*}
This simply means the $\bar{s}$ and $s^{tr}$ are of the same direction. By denoting it and its counterpart in eigen space as
\begin{gather*}
n=\dfrac{\bar{s}}{\norm{\bar{s}}}=\dfrac{s^{tr}}{\norm{s^{tr}}},\qquad{}\hat{n}=\dfrac{\hat{\bar{s}}}{\norm{\hat{\bar{s}}}}=\dfrac{\hat{s}^{tr}}{\norm{\hat{s}^{tr}}},
\end{gather*}
the effective deviatoric stress only depends on $\Delta\lambda$,
\begin{gather*}
\bar{s}=s^{tr}-2Gn\Delta\lambda.
\end{gather*}

In eigen space, the principal effective stress can be expressed as
\begin{gather*}
\hat{\bar{\sigma}}=\hat{\sigma}^{tr}-\Delta\lambda\left(2G\hat{n}+3K\alpha_pI\right).
\end{gather*}
Its derivative simply reads
\begin{gather}
\ddfrac{\hat{\bar{\sigma}}}{\Delta\lambda}=-\left(2G\hat{n}+3K\alpha_pI\right),
\end{gather}
which is a constant due to that $\hat{n}$ does not change.
\section{Flow Rule}
The flow rule used is a Drucker--Prager type function that defines the relationship between plastic strain and consistency parameter,
\begin{gather*}
\Delta\varepsilon^p=\Delta\lambda\ddfrac{G}{\bar{\sigma}}=\Delta\lambda\nabla{}G.
\end{gather*}
The function $G$ is chosen to be
\begin{gather*}
G=\sqrt{2\bar{J}_2}+\alpha_p\bar{I}_1.
\end{gather*}
Thus the plastic strain is expressed as
\begin{gather*}
\Delta\varepsilon^p=\Delta\lambda\left(n+\alpha_pI\right),
\end{gather*}
or in eigen space,
\begin{gather*}
\Delta\hat{\varepsilon}^p=\Delta\lambda\left(\hat{n}+\alpha_pI\right).
\end{gather*}

The second order derivative of $G$ is
\begin{gather}
\nabla^2G=\dfrac{I^d-n\otimes{}n}{\norm{\bar{s}}}.
\end{gather}
\section{Yield Function}
The yield function is chosen as
\begin{gather*}
F=\alpha\bar{I}_1+\sqrt{3\bar{J}_2}+\beta\left<\hat{\bar{\sigma}}_1\right>-\left(1-\alpha\right)c_c,
\end{gather*}
with
\begin{gather*}
\beta=\dfrac{c_c}{c_t}\left(1-\alpha\right)-\left(1+\alpha\right),
\end{gather*}
and
\begin{gather*}
c_c=-\bar{f}_c,\qquad{}c_t=\bar{f}_t.
\end{gather*}
Inserting those terms into the yield function leads to
\begin{gather*}
F=\alpha\bar{I}_1+\sqrt{3\bar{J}_2}+\beta\left<\hat{\bar{\sigma}}_1\right>+\left(1-\alpha\right)\bar{f}_c,\\\beta=-\dfrac{\bar{f}_c}{\bar{f}_t}\left(1-\alpha\right)-\left(1+\alpha\right).
\end{gather*}

Taking the derivatives gives
\begin{gather*}
\pdfrac{F}{\hat{\bar{\sigma}}}=\alpha{}I+\sqrt{\dfrac{3}{2}}\hat{n}+\beta{}H\left(\hat{\bar{\sigma}}_1\right)\pdfrac{\hat{\bar{\sigma}}_1}{\hat{\bar{\sigma}}},\\
\pdfrac{F}{\kappa}=\left(1-\alpha\right)\begin{bmatrix}
\dfrac{\left<\hat{\bar{\sigma}}_1\right>\bar{f}_c}{\bar{f}_t^2}\ddfrac{\bar{f}_t}{\kappa_t}\\[4mm]
\left(1-\dfrac{\left<\hat{\bar{\sigma}}_1\right>}{\bar{f}_t}\right)\ddfrac{\bar{f}_c}{\kappa_c}
\end{bmatrix}
\end{gather*}
\section{Damage Evolution}
The backbone curves are related to the damage parameter $\kappa_\aleph$.
\begin{gather*}
f_\aleph=f_{\aleph,0}\sqrt{\phi_\aleph}\Phi_\aleph,
\end{gather*}
with
\begin{gather*}
\phi_\aleph=1+a_\aleph\left(2+a_\aleph\right)\kappa_\aleph,\\
\Phi_\aleph=\dfrac{1+a_\aleph-\sqrt{\phi_\aleph}}{a_\aleph}.
\end{gather*}

The effective counterpart is
\begin{gather*}
\bar{f}_\aleph=\dfrac{f_\aleph}{1-d}=f_{\aleph,0}\sqrt{\phi_\aleph}\Phi_\aleph^{1-c_\aleph/b_\aleph},
\end{gather*}
with
\begin{gather*}
d=1-\Phi_\aleph^{c_\aleph/b_\aleph}.
\end{gather*}

The derivatives are
\begin{gather*}
\ddfrac{f_\aleph}{\kappa_\aleph}=f_{\aleph,0}\dfrac{a_\aleph+2}{2\sqrt{\phi_\aleph}}\left(a_\aleph-2\sqrt{\phi_\aleph}+1\right),\\
\ddfrac{\bar{f}_\aleph}{\kappa_\aleph}=f_{\aleph,0}\dfrac{a_\aleph+2}{2\sqrt{\phi_\aleph}}\dfrac{\left(a_\aleph+1+\left(\dfrac{c_\aleph}{b_\aleph}-2\right)\sqrt{\phi_\aleph}\right)}{\Phi_\aleph^{c_\aleph/b_\aleph}},\\
\ddfrac{d}{\kappa_\aleph}=\dfrac{c_\aleph}{b_\aleph}\dfrac{a_\aleph+2}{2\sqrt{\phi_\aleph}}\Phi_\aleph^{c_\aleph/b_\aleph-1}
\end{gather*}

The evolution of damage parameters can be written as
\begin{gather*}
\Delta\kappa=H\Delta\lambda,
\end{gather*}
The matrix $H$ reads
\begin{gather*}
H=\begin{bmatrix}
r\dfrac{f_t}{g_t}\left(\hat{n}_1+\alpha_p\right)\\\left(1-r\right)\dfrac{f_c}{g_c}\left(\hat{n}_3+\alpha_p\right)
\end{bmatrix},
\end{gather*}
and its derivatives are
\begin{gather*}
\pdfrac{H}{\kappa}=\text{diag}\begin{pmatrix}
\dfrac{r}{g_t}\left(\hat{n}_1+\alpha_p\right)\ddfrac{f_t}{\kappa_t}\\[4mm]
\dfrac{1-r}{g_c}\left(\hat{n}_3+\alpha_p\right)\ddfrac{f_c}{\kappa_c}
\end{pmatrix},\\
\pdfrac{H}{\hat{\bar{\sigma}}}=\begin{bmatrix}
\dfrac{f_t}{g_t}\left(\hat{n}_1+\alpha_p\right)\\
-\dfrac{f_c}{g_c}\left(\hat{n}_3+\alpha_p\right)
\end{bmatrix}\cdot\ddfrac{r}{\hat{\bar{\sigma}}}
\end{gather*}

The residual is expressed as
\begin{gather*}
Q=-\kappa_{n+1}+\kappa_n+\Delta\lambda{}H,
\end{gather*}
the total differentiation of which is
\begin{gather*}
\md{Q}=\pdfrac{Q}{\kappa}\md{\kappa}+\pdfrac{Q}{\hat{\bar{\sigma}}}\md{\hat{\bar{\sigma}}}+\pdfrac{Q}{\Delta\lambda}\md{\Delta\lambda},
\end{gather*}
with
\begin{gather*}
\pdfrac{Q}{\kappa}=\Delta\lambda\pdfrac{H}{\kappa}-I,\quad
\pdfrac{Q}{\hat{\bar{\sigma}}}=\Delta\lambda\pdfrac{H}{\hat{\bar{\sigma}}},\quad
\pdfrac{Q}{\Delta\lambda}=H.
\end{gather*}

From yield function, we have
\begin{gather*}
\pdfrac{F}{\hat{\bar{\sigma}}}\ddfrac{\hat{\bar{\sigma}}}{\Delta\lambda}\md{\Delta\lambda}+\pdfrac{F}{\kappa}\md{\kappa}=0,
\end{gather*}
so
\begin{gather*}
\md{\Delta\lambda}=-\dfrac{1}{\pdfrac{F}{\hat{\bar{\sigma}}}\ddfrac{\hat{\bar{\sigma}}}{\Delta\lambda}}\pdfrac{F}{\kappa}\md{\kappa},
\end{gather*}
insert in residual expression,
\begin{gather*}
\ddfrac{Q}{\kappa}=\pdfrac{Q}{\kappa}-\dfrac{\pdfrac{Q}{\hat{\bar{\sigma}}}\ddfrac{\hat{\bar{\sigma}}}{\Delta\lambda}+\pdfrac{Q}{\Delta\lambda}}{\pdfrac{F}{\hat{\bar{\sigma}}}\ddfrac{\hat{\bar{\sigma}}}{\Delta\lambda}}\pdfrac{F}{\kappa},
\end{gather*}
which is equivalent to
\begin{gather}
\ddfrac{Q}{\kappa}=\Delta\lambda\pdfrac{H}{\kappa}-I-\dfrac{\Delta\lambda\pdfrac{H}{\hat{\bar{\sigma}}}\ddfrac{\hat{\bar{\sigma}}}{\Delta\lambda}+H}{\pdfrac{F}{\hat{\bar{\sigma}}}\ddfrac{\hat{\bar{\sigma}}}{\Delta\lambda}}\pdfrac{F}{\kappa},
\end{gather}
\section{Algorithm}
Collecting all terms, the yield function is
\begin{multline*}
\alpha{}I_1^{tr}-9K\alpha\alpha_p\Delta\lambda+\sqrt{\dfrac{3}{2}}\left(\norm{s^{tr}}-2G\Delta\lambda\right)\\+\beta\left<\hat{\sigma}^{tr}_1-\Delta\lambda\left(2G\hat{n}_1+3K\alpha_p\right)\right>+\left(1-\alpha\right)\bar{f}_c=0,
\end{multline*}
The increment of $\lambda$ can be expressed as
\begin{gather}
\Delta\lambda=\dfrac{\alpha{}I_1^{tr}+\sqrt{3J_2^{tr}}+\left<\beta\right>\hat{\sigma}^{tr}_1+\left(1-\alpha\right)\bar{f}_c}{9K\alpha\alpha_p+\sqrt{6}G+\left<\beta\right>\left(2G\hat{n}_1+3K\alpha_p\right)}
\end{gather}
\section{Stiffness}
The consistent tangent stiffness can be derived from three converged equations, which are the stress updating
\begin{gather*}
\sigma=\left(1-d\right)\bar{\sigma}=\left(1-d\right)E\left(\varepsilon-\varepsilon^p\right),
\end{gather*}
and yield function $F=0$ and residual of the evolution of damage parameters $Q=0$. Differentiating these two equations gives
\begin{gather*}
\pdfrac{F}{\hat{\bar{\sigma}}}\md{\hat{\bar{\sigma}}}+\pdfrac{F}{\kappa}\md{\kappa}=0,\\
\pdfrac{Q}{\kappa}\md{\kappa}+\pdfrac{Q}{\Delta\lambda}\md{\Delta\lambda}+\pdfrac{Q}{\hat{\bar{\sigma}}}\md{\hat{\bar{\sigma}}}=0.
\end{gather*}

The total differentiation of the final stress can be expressed as
\begin{equation}
\begin{split}
\md{\sigma}&=-\bar{\sigma}\md{d}+\left(1-d\right)\md{\bar{\sigma}}\\
&=-\bar{\sigma}\left(\pdfrac{d}{\kappa}\md{\kappa}+\pdfrac{d}{r}\ddfrac{r}{\hat{\bar{\sigma}}}\ddfrac{\hat{\bar{\sigma}}}{\bar{\sigma}}\md{\bar{\sigma}}\right)+\left(1-d\right)\md{\bar{\sigma}}.
\end{split}
\end{equation}

Since $d=1-\left(1-d_c\right)\left(1-sd_t\right)$ with $s=s_0+\left(1-s_0\right)r$, we have
\begin{gather*}
1-d=\left(1-d_c\right)\left(1-sd_t\right),\\
\pdfrac{d}{\kappa}=\begin{bmatrix}
\left(s-sd_c\right)\ddfrac{d_t}{\kappa_t}&\left(1-sd_t\right)\ddfrac{d_c}{\kappa_c}
\end{bmatrix},\\
\pdfrac{d}{r}=\pdfrac{d}{s}\ddfrac{s}{r}=d_t\left(1-d_c\right)\left(1-s_0\right).
\end{gather*}

Now we focus on the effective part $\md{\bar{\sigma}}$.
\begin{gather*}
E^{-1}\md{\bar{\sigma}}=\md{\varepsilon}-\md{\varepsilon^p}.
\end{gather*}
Since $\varepsilon^p=\Delta\lambda{}\nabla{}G$,
\begin{gather*}
E^{-1}\md{\bar{\sigma}}=\md{\varepsilon}-\nabla{}G\md{\Delta\lambda}-\Delta\lambda\md{\nabla{}G},\\
\left(E^{-1}+\Delta\lambda\nabla^2G\right)\md{\bar{\sigma}}=\md{\varepsilon}-\nabla{}G\md{\Delta\lambda},
\end{gather*}
denoting
\begin{gather}
S=\left(E^{-1}+\Delta\lambda\dfrac{I^d-n\otimes{}n}{\norm{\bar{s}}}\right)^{-1},
\end{gather}
where $I^d$ is the deviatoric unit fourth--order tensor,
\begin{gather*}
\md{\bar{\sigma}}=S\md{\varepsilon}-S\nabla{}G\md{\Delta\lambda},
\end{gather*}

From the residual,
\begin{gather*}
\md{\kappa}=-\pdfrac{Q}{\kappa}^{-1}\left(\pdfrac{Q}{\Delta\lambda}\md{\Delta\lambda}+\pdfrac{Q}{\hat{\bar{\sigma}}}\md{\hat{\bar{\sigma}}}\right),
\end{gather*}
insert it into the yield function,
\begin{gather*}
\pdfrac{F}{\hat{\bar{\sigma}}}\md{\hat{\bar{\sigma}}}=\pdfrac{F}{\kappa}\pdfrac{Q}{\kappa}^{-1}\left(\pdfrac{Q}{\Delta\lambda}\md{\Delta\lambda}+\pdfrac{Q}{\hat{\bar{\sigma}}}\md{\hat{\bar{\sigma}}}\right),
\end{gather*}
rearranging gives
\begin{gather*}
\left(\pdfrac{F}{\hat{\bar{\sigma}}}-\pdfrac{F}{\kappa}\pdfrac{Q}{\kappa}^{-1}\pdfrac{Q}{\hat{\bar{\sigma}}}\right)\ddfrac{\hat{\bar{\sigma}}}{\bar{\sigma}}\md{\bar{\sigma}}=\pdfrac{F}{\kappa}\pdfrac{Q}{\kappa}^{-1}\pdfrac{Q}{\Delta\lambda}\md{\Delta\lambda},
\end{gather*}
denote
\begin{gather}
T=\left(\pdfrac{F}{\hat{\bar{\sigma}}}-\pdfrac{F}{\kappa}\pdfrac{Q}{\kappa}^{-1}\pdfrac{Q}{\hat{\bar{\sigma}}}\right)\ddfrac{\hat{\bar{\sigma}}}{\bar{\sigma}},
\end{gather}
we have
\begin{gather*}
T\left(S\md{\varepsilon}-S\nabla{}G\md{\Delta\lambda}\right)=\pdfrac{F}{\kappa}\pdfrac{Q}{\kappa}^{-1}\pdfrac{Q}{\Delta\lambda}\md{\Delta\lambda},
\end{gather*}
the $\md{\Delta\lambda}$ can be then solved,
\begin{gather}
\ddfrac{\Delta\lambda}{\varepsilon}=\dfrac{TS}{TS\nabla{}G+\pdfrac{F}{\kappa}\pdfrac{Q}{\kappa}^{-1}\pdfrac{Q}{\Delta\lambda}}.
\end{gather}
Finally, the $\md{\bar{\sigma}}$ can be expressed as a function of $\md{\varepsilon}$,
\begin{gather}
\ddfrac{\bar{\sigma}}{\varepsilon}=S-S\nabla{}G\ddfrac{\Delta\lambda}{\varepsilon},
\end{gather}

The final differentiation can be expressed as
\begin{multline}
\ddfrac{\sigma}{\varepsilon}=\bar{\sigma}\pdfrac{d}{\kappa}\pdfrac{Q}{\kappa}^{-1}\pdfrac{Q}{\Delta\lambda}\ddfrac{\Delta\lambda}{\varepsilon}+\\\left(\left(1-d\right)I+\bar{\sigma}\left(\pdfrac{d}{\kappa}\pdfrac{Q}{\kappa}^{-1}\pdfrac{Q}{\hat{\bar{\sigma}}}-\pdfrac{d}{\hat{\bar{\sigma}}}\right)\ddfrac{\hat{\bar{\sigma}}}{\bar{\sigma}}\right)\ddfrac{\bar{\sigma}}{\varepsilon}.
\end{multline}
\end{document}
